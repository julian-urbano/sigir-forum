% CHOOSE A PUBLICATION TYPE
\documentclass[news]{sigirforum}
%\documentclass[keynote]{sigirforum}
%\documentclass[paper]{sigirforum}
%\documentclass[event]{sigirforum}
%\documentclass[opinion]{sigirforum}
%\documentclass[dissertation]{sigirforum}

% OVERWRITES FOR THE INSTRUCTIONS ONLY
\def\pubtype{INSTRUCTIONS}
% DEFINE METADATA FOR VOLUME AND NUMBER
\def\pubissue{v1.0 -- February 2022}

\begin{document}
% USE TITLE CASE FOR THE TITLE
\title{\LaTeX\xspace Class for the SIGIR Forum}

\authors{
Tirthankar Ghosal, Josiane Mothe, Juli\'an Urbano\\
SIGIR Forum editors\\
\small{\texttt{editors\_SIGIR@acm.org}}}

\maketitle 
\begin{abstract}
This document contains the instructions for the SIGIR Forum publication template in \LaTeX\xspace as of version 1.0, February 2022. For the latest version, please visit the Github repository.\footnote{\url{https://github.com/julian-urbano/sigir-forum}}
\end{abstract}

\section{Publication Type}
%%%%%%%%%%%%%%%%%%%%%%%%%%%%%%%%%%%%%%%%%%%%%%%%%%%%%%%%%%%%%%%%%%%%%%%%%%%%%%%%

The publication type is indicated as an option to the class, such as
\begin{verbatim}
  \documentclass[news]{sigirforum}
\end{verbatim}
Currently supported types are:
\begin{itemize}
	\item \texttt{news} for publications about the SIG, such as the Chair's letter or the annual business meeting notes.
	\item \texttt{keynote} for the extended abstract of a keynote presented at a relevant event.
	\item \texttt{paper} for a research paper.
	\item \texttt{event} for the report about a relevant event, such as a conference or workshop.
	\item \texttt{opinion} for an opinion paper.
	\item \texttt{dissertation} for a PhD dissertation abstract.
\end{itemize}

If your publication does not fit in these categories, please contact the editors.

\section{Title}
%%%%%%%%%%%%%%%%%%%%%%%%%%%%%%%%%%%%%%%%%%%%%%%%%%%%%%%%%%%%%%%%%%%%%%%%%%%%%%%%

The title of the publication should follow Title Case and is entirely up to the authors, but there are exceptions. In \textbf{event reports}, please follow this template:
\begin{enumerate}
	\item Main-event: \textsl{``Report on the 1st Conference on Foo (Foo 2021)''}.
	\item Co-located event: \textsl{``Report on the 1st Workshop on Bar (Bar 2021) at Foo 2021''}.
	\item Special sessions or panels: \textsl{``Report on the Foo 2021 special session on Bar''}.
\end{enumerate}

If your event does not fit in these categories, please contact the editors for suggestions. In \textbf{keynote abstracts}, please follow this template: \textsl{``The Title: A keynote at Foo 2021''}. In \textbf{dissertation abstracts}, please use the same title as in your dissertation.

\section{Authors}
%%%%%%%%%%%%%%%%%%%%%%%%%%%%%%%%%%%%%%%%%%%%%%%%%%%%%%%%%%%%%%%%%%%%%%%%%%%%%%%%

All authors must have an affiliation including country, and at least one of them must indicate a contact email address. Authors are specified with commands \texttt{\textbackslash authors} and \texttt{\textbackslash author}, as follows:
\begin{verbatim}
  \authors{
    \author[jane.doe@sigir.org]{Jane Doe}{ACM SIGIR}{USA}
    \and
    \author{John Doe}{ACM SIGIR}{USA}
    % \and ...
  }
\end{verbatim}
Note how e-mail addresses are indicated as an optional parameter to \texttt{\textbackslash author}.

In \textbf{dissertation abstracts}, only the author of the dissertation will appear in the SIGIR Forum publication. Please give your current affiliation; the institution granting the doctorate will appear in the metadata.

\section{Metadata}
%%%%%%%%%%%%%%%%%%%%%%%%%%%%%%%%%%%%%%%%%%%%%%%%%%%%%%%%%%%%%%%%%%%%%%%%%%%%%%%%

Update the volume and number metadata in the preamble:
\begin{verbatim}
  \def\pubissue{Vol. 55 No. 2 December 2021}
\end{verbatim}
If unsure, please contact the editors.

Some publications have special formatting for metadata. In \textbf{event reports}, please indicate in the preamble the date(s) of the event and the URL of its website:
\begin{verbatim}
  \def\eventdate{10--14 September, 2021}
  \def\eventurl{https://foo2021.net}
\end{verbatim}
If your event does not have a website, leave \texttt{\textbackslash eventurl} empty:
\begin{verbatim}
  \def\eventurl{}
\end{verbatim}

In \textbf{dissertation abstracts}, please indicate the institution that granted the degree, the defense date, the name(s) of the supervisor(s) without titles (ie. no \textsl{Dr.} or \textsl{Prof.}), and a URL to access the dissertation:
\begin{verbatim}
  \def\dissertationplace{University of SIGIR, City, Country}
  \def\dissertationdate{1 January 1970}
  \def\dissertationsupervisor{Leslie Lamport}
  \def\dissertationurl{https://name.domain/thesis-file.pdf}
\end{verbatim}
If there is no website, leave it empty:
\begin{verbatim}
  \def\dissertationurl{}
\end{verbatim}

\section{Text}
%%%%%%%%%%%%%%%%%%%%%%%%%%%%%%%%%%%%%%%%%%%%%%%%%%%%%%%%%%%%%%%%%%%%%%%%%%%%%%%%

In \textbf{disseration abstracts}, please keep all the text within the \texttt{abstract} environment; no text is allowed out of it. For other publication types, write an abstract and use sections as usual. There is no page limit, except for \textbf{dissertation abstracts}, which have a limit of one page plus references and metadata.

Please refrain from using \href{https://sigir.org}{hidden links like this}. All in-text URLs must be written with the \texttt{\textbackslash url} command, and they should appear in footnotes whenever possible. Exceptions are possible if adding footnotes would harm readability. Please contact the editors if unsure.

%It is possible for a URL to span over multiple lines or run over the margins. Please split these multi-line URLs with \texttt{\textbackslash href} and \texttt{\textbackslash linebreak}, as follows:
%\begin{verbatim}
%  \href{<full-url>}
%  {\texttt{<1st-line>}\linebreak
%   \texttt{<2nd-line>}\linebreak
%   ...}
%\end{verbatim}
%
%This will make long URLs\footnote{\url{https://very.long.domain.com/this-is-a-very-long-url-than-runs-over-the-margin-so-we-should-split-it-manually.html}} span over multiple lines and retain the correct hyperlink, such as:\footnote{\href{https://very.long.domain.com/this-is-a-very-long-url-than-runs-over-the-margin-so-we-should-split-it-manually.html}{\texttt{https://very.long.domain.com/this-is-a-very-long-url-than-runs-over-the-margin-}\linebreak\texttt{so-we-should-split-it-manually.html}}}
%\begin{verbatim}
%  \href
%  {https://very.long.domain.com/this-is-a-very-long-url-than-runs-over-the-margin-
%   so-we-should-split-it-manually.html}
%  {\texttt{https://domain.com/this-is-a-very-long-url-than-runs-over-the-margin-}
%   \linebreak\texttt{so-we-should-split-it-manually.html}}
%\end{verbatim}

Please also refrain from using \texttt{\textbackslash paragraph}. Consider using (sub$^n$)sections or lists.

\section{References and Citations}
%%%%%%%%%%%%%%%%%%%%%%%%%%%%%%%%%%%%%%%%%%%%%%%%%%%%%%%%%%%%%%%%%%%%%%%%%%%%%%%%

Please use citations with package \texttt{natbib}\footnote{\url{https://ctan.org/pkg/natbib?lang=en}}.
You can use parenthetical citations like~\citep{forum} through \texttt{\textbackslash citep}, or textual citations like~\citet{forum} through \texttt{\textbackslash citet}.

In \textbf{dissertation abstracts} you can only reference up to five of your own publications, but not someone else's. If you would rather just list the publications but not cite them in the abstract, use command \texttt{\textbackslash nocite}.

Please use a \texttt{.bib} file for the references; do not manually include the \texttt{\textbackslash bibitem}s in the \texttt{.tex} source.

\section{Acknowledgments}
%%%%%%%%%%%%%%%%%%%%%%%%%%%%%%%%%%%%%%%%%%%%%%%%%%%%%%%%%%%%%%%%%%%%%%%%%%%%%%%%

Acknowledgments should appear at the end of the manuscript, using \texttt{\textbackslash section*\{Acknowledgments\}}.

\section{Submission}
%%%%%%%%%%%%%%%%%%%%%%%%%%%%%%%%%%%%%%%%%%%%%%%%%%%%%%%%%%%%%%%%%%%%%%%%%%%%%%%%

You can submit your publication via email to \url{editors\_SIGIR@acm.org}. Please include all the source files and a compiled PDF.

Attached to these instructions there is a \textbf{checklist document} with a list of items to check that your manuscript follows the SIGIR Forum template. Please check all items before you make a submission.

\bibliography{sigirforum}
\end{document}
